Much of computer vision has been devoted to the question of representation
through feature extraction. Useful features transform the raw pixel intensity
values to a representation in which common problems such as identification,
tracking, and segmentation of objects are easier to solve. Recently, deep
feature hierarchies have proven to be immensely successful at solving many
problems in computer vision. In the supervised setting, these hierarchies are
trained to solve specific problems by minimizing an objective function of the
data and problem specific label information. Recent findings suggest that
despite being trained on a specific task, the learned features can be
transferred across multiple visual tasks. These findings suggests that there
exists a generically useful feature representation for natural visual data.    

This work aims to uncover the principles that lead to these generic feature
representations in the unsupervised setting, which does not require problem
specific label information. We begin by reviewing relevant prior work,
particularly the literature on auto-encoder networks and energy based learning.
We introduce a new regularizer for auto-encoders that plays an analogous role
to the partition function in probabilistic graphical models.  Next we explore
the role of specialized encoder architectures for sparse inference. The
remainder of the thesis explores visual feature learning from video. We
establish a connection between slow-feature learning and metric learning, and
experimentally demonstrate that semantically coherent metrics can be learned
from natural videos. Finally, we posit that useful features linearize
natural image transformations in video. To this end, we introduce a new
architecture and loss for training deep feature hierarchies that linearize the
transformations observed in unlabeled natural video sequences by learning to
predict future frames in the presence of uncertainty.           


 
